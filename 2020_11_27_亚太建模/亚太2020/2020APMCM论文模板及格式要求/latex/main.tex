% !Mode:: "TeX:UTF-8"
%%  本模板推荐以下方式编译:
%%
%%     1. PDFLaTeX[推荐]
%%  注意:
%%   1. 文件默认的编码为 UTF-8 对于windows,请选用支持UTF-8编码的编辑器。
%%   2. 若是模板有什么问题,请及时与我们取得联系,Email:latexstudio@qq.com。
%%   3. 可以到  https://wenda.latexstudio.net 提问
%%   4. 请安装 最新版本的 TeXLive 地址:https://www.latexstudio.net/page/texsoftware/

\documentclass{apmcmthesis}

\usepackage{url}

%%%%%%%%%%%%填写相关信息%%%%%%%%%%%%%%%%%%%%%%%%%%
\tihao{A}                            %选题
\baominghao{2020XXXXXXXXX}           %报名号
\begin{document}

\pagestyle{frontmatterstyle}

\begin{abstract}
\verb|apmcmthesis| \LaTeX{} template is designed by \url{https://www.latexstudio.net} for \url{http://www.apmcm.org}. The template is designed to let everyone focus on the content writing of the paper, without spending too much effort on the customization and adjustment of the format.

Note that users need to have some experience with \LaTeX{}, at least some of the features of common macro packages, such as references, math formulas, image usage, list environment, etc. Templates have added commonly used macros. Package, no additional user added.

This template is located on   \url{https://github.com/latexstudio/APMCMThesis}. You can update files from the repository.


\keywords{Keywords1\quad  Keywords2\quad   Keywords3}
\end{abstract}



\newpage
%目录
\tableofcontents


\newpage
\pagestyle{mainmatterstyle}
\setcounter{page}{1}
\section{Introduction}
In order to indicate the origin of problems, the following background is worth mentioning.
\subsection{}


\subsection{}



\subsection{}


\section{The Description of the Problem}
\subsection{How do we approximate the whole course of ?}

\begin{itemize}
  \item
  \item
  \item
\end{itemize}


\subsection{How do we define the optimal configuration?}
1) From the perspective of      :\par
2) From the perspective of the      :\par
3) Compromise:

\subsection{The local optimization and the overall optimization}


\begin{itemize}
  \item
  \item
  \item Virtually:
\end{itemize}


\subsection{The differences in weights and sizes of}


\subsection{What if there is no data available?}






\section{Models}
\subsection{Basic Model}


\subsubsection{Terms, Definitions and Symbols}
The signs and definitions are mostly generated from queuing theory.


\subsubsection{Assumptions}


\subsubsection{The Foundation of Model}
1) The utility function

\begin{itemize}
\item The cost of       :
\item The loss of       :
\item The weight of each aspect:
\item Compromise:
\end{itemize}



\begin{figure}[!ht]
  \centering
  \includegraphics[width=3cm]{cat}  \quad  \includegraphics[width=5cm]{gongzhonghao}
  \caption{\CJK{UTF8}{gbsn}{关注我们公众号,学习更多知识}}\label{cat1}
\end{figure}



3) The overall optimization and the local optimization

\begin{itemize}
\item The overall optimization:
\item The local optimization:
\item The optimal number of        :
\end{itemize}



\subsubsection{Solution and Result}
1) The solution of the integer programming:
2) Results:
\subsubsection{Analysis of the Result}
\begin{itemize}
\item Local optimization and overall optimization:
\item Sensitivity: The result is quite sensitive to the change of the three parameters
\item
\item Trend:
\item Comparison:
\end{itemize}
\subsubsection{Strength and Weakness}

\begin{description}
\item[Strength:] The Improved Model aims to make up for the neglect of         . The result seems to declare that this model is more reasonable than the Basic Model and much more effective than the existing design.
\item[Weakness:] Thus the model is still an approximate on a large scale. This has doomed to limit the applications of it.
\end{description}

\section{Conclusions}

\subsection{Conclusions of the problem}
\begin{itemize}
\item 	
\item
\item
\item
\end{itemize}	
\subsection{Methods used in our models}
\begin{itemize}
\item 	
\item
\item
\item
\end{itemize}
\subsection{Applications of our models}
\begin{itemize}
\item 	
\item
\item
\item
\end{itemize}
\section{Future Work}
\subsection{Another model}
\subsubsection{The limitations of queuing theory}




\subsubsection{}


\subsubsection{}



\subsubsection{}





%参考文献
\begin{thebibliography}{9}%宽度9
\bibitem{1} Author, Title, Place of Publication: Press, Year of publication.
\bibitem{2} author, paper name, magazine name, volume number: starting and ending
page number, year of publication.
\bibitem{3} author, resource title, web site, visit time (year, month, day).
 \bibitem{bib:one} \LaTeX{}\CJK{UTF8}{gbsn}{资源和技巧学习} \url{https://www.latexstudio.net}
 \bibitem{bib:two} \LaTeX{}\CJK{UTF8}{gbsn}{问题交流网站} \url{https://wenda.latexstudio.net}
  \bibitem{bib:two} \CJK{UTF8}{gbsn}{模板库维护} \url{https://github.com/latexstudio/APMCMThesis}
\end{thebibliography}

\newpage
%附录

\section{Appendix}
\begin{lstlisting}[language=matlab,caption={The matlab Source code of Algorithm}]
kk=2;[mdd,ndd]=size(dd);
while ~isempty(V)
[tmpd,j]=min(W(i,V));tmpj=V(j);
for k=2:ndd
[tmp1,jj]=min(dd(1,k)+W(dd(2,k),V));
tmp2=V(jj);tt(k-1,:)=[tmp1,tmp2,jj];
end
tmp=[tmpd,tmpj,j;tt];[tmp3,tmp4]=min(tmp(:,1));
if tmp3==tmpd, ss(1:2,kk)=[i;tmp(tmp4,2)];
else,tmp5=find(ss(:,tmp4)~=0);tmp6=length(tmp5);
if dd(2,tmp4)==ss(tmp6,tmp4)
ss(1:tmp6+1,kk)=[ss(tmp5,tmp4);tmp(tmp4,2)];
else, ss(1:3,kk)=[i;dd(2,tmp4);tmp(tmp4,2)];
end;end
dd=[dd,[tmp3;tmp(tmp4,2)]];V(tmp(tmp4,3))=[];
[mdd,ndd]=size(dd);kk=kk+1;
end; S=ss; D=dd(1,:);
 \end{lstlisting}
\begin{lstlisting}[language=c,caption={The lingo source code}]
kk=2;
[mdd,ndd]=size(dd);
while ~isempty(V)
    [tmpd,j]=min(W(i,V));tmpj=V(j);
for k=2:ndd
    [tmp1,jj]=min(dd(1,k)+W(dd(2,k),V));
    tmp2=V(jj);tt(k-1,:)=[tmp1,tmp2,jj];
end
    tmp=[tmpd,tmpj,j;tt];[tmp3,tmp4]=min(tmp(:,1));
if tmp3==tmpd, ss(1:2,kk)=[i;tmp(tmp4,2)];
else,tmp5=find(ss(:,tmp4)~=0);tmp6=length(tmp5);
if dd(2,tmp4)==ss(tmp6,tmp4)
    ss(1:tmp6+1,kk)=[ss(tmp5,tmp4);tmp(tmp4,2)];
else, ss(1:3,kk)=[i;dd(2,tmp4);tmp(tmp4,2)];
end;
end
    dd=[dd,[tmp3;tmp(tmp4,2)]];V(tmp(tmp4,3))=[];
    [mdd,ndd]=size(dd);
    kk=kk+1;
end;
S=ss;
D=dd(1,:);
 \end{lstlisting}


\end{document} 